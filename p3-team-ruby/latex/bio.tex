%\documentclass[sigconf, authordraft]{}
\documentclass[a4paper,10pt]{article}
\usepackage[utf8]{inputenc}
\usepackage{graphicx}
\usepackage{float}

\newcommand{\biff}[0]{\textbf{Biff}}


%opening
\title{Team Ruby: Social Media App}
\author {Annika Dame (annikadame@u.boisestate.edu), \\
Jaden Dawdy (jadendawdy@u.boisestate.edu), \\
Camilla Eckhardt (camillaeckhardt@u.boisestate.edu), \\
Xian Ma (xianhuama@u.boisestate.edu)}


\begin{document}

  \maketitle

  \section{Introduction}
  \label{sec:into}

  Ruby is a very powerful and widely used language today. It is foundational in many large codebases and acts as the backend for
  many widely used websites such as Twitter and Shopify. We noticed the unique set of syntax that was akin to Python which furthered our interest
  in the language. Furthermore, the language seems to lend itself to many use cases outside of web applications such as data science and scripting.

  Our motivation for the coding portion stems from us noticing that were a lot of key features that social media platforms
  did not share. We wanted to take the best of different platforms and create one with all the noted key features that we thought were essential.
  Additionally, we found Ruby to be very versatile in its applications in the realm of web development. The logo of the language is shown in Figure \ref{fig:ruby_logo}.

    \begin{figure}
  \centering
      \includegraphics[width=.25\columnwidth]{ruby_logo.png}
      \caption{Ruby Logo - Yukihiro Matsumoto}
      \label{fig:ruby_logo}
  \end{figure}


  \cite{RubyLogo}


  \section{Language Overview Proposal}
  \label{sec: overview_proposal}

  \subsection{Language Overview}

  The language overview will be done through a detailed presentation that comprises of a multitude of parts. A portion of the explanation of the
  language will be done through the demoing of simple intro programs that demonstrate the various features of Ruby alongside how to install and setup the language.
  The presentation will also cover the the history of the language and its description. Sprinkled throughout the presentation will be how Ruby relates to or comprises of
  the concepts that have been covered in class in regards to programming languages.

  \section{Programming Assignment Proposal}
  \label{sec: programming_proposal}

  \subsection{The Program}

  The program that we chose to create with Ruby is a social media application that aims to take the key features of many social media platforms and
  implement them into one singular application. The program will be able to log user interaction and user posts. The user posts will be comprised of either
  images, text, or both at the same time. The user interaction will consist of "likes" and comments.

  \subsection{Program Evaluation}

  Since this program will focus a large part on user interaction, we will use mainly user experience testing for the front end. The functionality of the user interaction
  will be measured through unit testing. The backend portion of the web application will be tested through unit tests. We will aim to utilize a Ruby testing framework.

  \subsection{Project Management}

  We plan to as a group get familiar with the Ruby On Rails (refer to \cite{RubyOnRails}) framework in the first week. We want to understand the many features and nuiances
  that the Ruby language offers alongside its frameworks. The following week, the programming group will aim to create functionalities with the web application
  utilizing the Ruby On Rails framework. Concurrently or in parallel the research group will start doing research on the Ruby programming language.
  The gathered research will be about the many aforementioned aspects that will covered in our presentation.

  \section{Presentation Proposal}
  \label{sec: presentation_proposal}

  The presentation will be half of the language overview and half demo of the program. The language group that will go over the language overview, consisting of Annika Dame
  and Camilla Eckhardt. The group will aim to cover the key findings from our research. The programming group that will go over the demo of the program, consisting of
  Jaden Dawdy and Xian Ma. This group will cover the features and functionalities that were created using Ruby.

% ACM-Reference-Format
%\bibliographystyle{chicago}
\bibliographystyle{plainnat}
\bibliography{refs}

\end{document}
